\documentclass[a4paper,12pt]{article}
\usepackage{amsmath, amssymb, graphicx, hyperref}
\usepackage[T1]{fontenc}
\usepackage{geometry}
\geometry{margin=1in}

\title{Descrizione dettagliata dell'algoritmo \texttt{minover}}
\author{Autore}
\date{\today}

\begin{document}

\maketitle

\section{Introduzione}
L'algoritmo \texttt{minover} \`e un metodo iterativo per trovare un punto all'interno di un politopo definito da un insieme di vincoli lineari. L'algoritmo cerca il vincolo pi\`u violato e sposta la soluzione corrente in direzione per soddisfarlo, minimizzando lo scostamento rispetto ai limiti imposti.\newline
Questo documento fornisce una descrizione dettagliata dell'algoritmo, analizzando ogni variabile e il suo ruolo nel processo.

\section{Descrizione dell'algoritmo}
L'algoritmo \texttt{minover} segue questi passaggi:

\begin{enumerate}
    \item Identificare il vincolo pi\`u violato, ovvero quello per cui la soluzione corrente \`e pi\`u lontana dai limiti imposti.
    \item Calcolare un fattore di correzione per riportare la soluzione entro i limiti.
    \item Applicare l'aggiornamento alla soluzione corrente.
    \item Ripetere il processo finch\'e tutti i vincoli sono soddisfatti.
\end{enumerate}

\section{Definizione delle variabili}
Le seguenti variabili sono utilizzate nell'implementazione:

\begin{itemize}
    \item \textbf{ok} ($\in \{0,1\}$): Flag che indica se tutti i vincoli sono soddisfatti ($ok=1$) o meno ($ok=0$).
    \item \textbf{alpha} ($\alpha$): Fattore di aggiornamento utilizzato per correggere la soluzione.
    \item \textbf{xmin}: Il valore minimo tra le violazioni dei vincoli.
    \item \textbf{min}: Indice del vincolo pi\`u violato.
    \item \textbf{sign} ($\pm1$): Indica se la correzione avviene verso l'alto o verso il basso.
    \item \textbf{flux}: Vettore delle variabili della soluzione corrente.
    \item \textbf{matrice}: Lista che memorizza gli indici delle variabili associate a ciascun vincolo.
    \item \textbf{matrice2}: Lista dei coefficienti dei vincoli.
    \item \textbf{boundmin, boundmax}: Limiti inferiore e superiore per ciascun vincolo.
    \item \textbf{norm}: Norma della direzione di aggiornamento.
\end{itemize}

\section{Passaggi dettagliati}
L'algoritmo esegue un ciclo finch\'e non trova una soluzione valida:

\begin{enumerate}
    \item Inizialmente, imposta $xmin$ ad un valore molto grande per identificare il vincolo pi\`u violato.
    \item Per ogni vincolo $i$, calcola il valore di $x$ dato dalla combinazione lineare:
    \begin{equation}
        x = \sum_{j} \text{flux}[\text{matrice}[i][j]] \cdot \text{matrice2}[i][j]
    \end{equation}
    \item Calcola le distanze dai limiti inferiori e superiori:
    \begin{equation}
        x_1 = x - \text{boundmin}[i], \quad x_2 = \text{boundmax}[i] - x
    \end{equation}
    \item Se $x_1$ o $x_2$ sono i minimi rilevati, aggiorna $xmin$, $min$ e $sign$.
    \item Se $xmin > 0$, termina l'algoritmo con $ok=1$.
    \item Altrimenti, calcola la norma della direzione di aggiornamento:
    \begin{equation}
        \text{norm} = \sum_{j} \text{matrice2}[\text{min}][j]^2
    \end{equation}
    \item Aggiorna la soluzione corrente con:
    \begin{equation}
        \alpha = -1.8 \frac{xmin}{\text{norm}}, \quad \text{flux}[\text{matrice}[min][j]] += \text{sign} \cdot \alpha \cdot \text{matrice2}[min][j]
    \end{equation}
    \item Continua finch\'e tutti i vincoli sono soddisfatti.
\end{enumerate}

\section{Conclusioni}
L'algoritmo \texttt{minover} \`e un metodo efficace per trovare un punto interno a un politopo rispettando i vincoli. L'approccio iterativo consente di correggere iterativamente le violazioni, rendendolo un metodo robusto per problemi di ottimizzazione con vincoli lineari.

\end{document}


% \documentclass[11pt, letterpaper]{article}

% % -------------------------------------------------------------------------
% % Commands
% \usepackage[utf8]{inputenc}
% \usepackage{graphicx}
% \usepackage{hyperref}
% \usepackage{cite}
% \usepackage{vmargin}
% \usepackage{lipsum}
% \usepackage{amsmath}

% \setmargins{2.5cm}                 % margen izquierdo
% {1.5cm}                            % margen superior
% {16.5cm}                           % anchura del texto
% {23.42cm}                          % altura del texto
% {10pt}                             % altura de los encabezados
% {1cm}                              % espacio entre el texto y los encabezados
% {0pt}                              % altura del pie de página
% {2cm}                              % espacio entre el texto y el pie de página

% % \hyphenation{}

% % \hypersetup{colorlinks,%
% % 	citecolor=blue,%s
% % 	filecolor=blue,%
% % 	linkcolor=blue,%
% % 	urlcolor=blue%
% % }


% % -------------------------------------------------------------------------
% \title{\textbf{$U_i$ dynamics}}

% % -------------------------------------------------------------------------

% \date{}
% \author{
%   % \textbf{Jos\'e A. Pereiro-Morej\'on$^{1,3}$, Roberto Mulet$^{1,2}$} \\
%   % \small{1. Group of Complex Systems and Statistical Physics.} \\
%   % \small{Physics Faculty, University of Havana, CP 10400. La Habana, Cuba} \\
%   % \small{2. Department of Theoretical Physics, }\\
%   % \small{Physics Faculty Faculty, University of Havana, CP 10400. La Habana, Cuba} \\
%   % \small{3. Biology Faculty, University of Havana, CP 10400. La Habana, Cuba} \\
% }

% % -------------------------------------------------------------------------
% % opening
% \begin{document}

% \maketitle

% \begin{abstract}
% 	\lipsum[2]
% \end{abstract}

% Baajskhdbck asdckjs dlajdbcliasd claksdc kjhaskdc \cite{pereiro-morejonInferenceMetabolicFluxes2022}. 
% jshdkasjdh fkashdvf kasjdf, such as:

% \begin{equation}
%   U_i(t+1) = (1-\lambda)U_i(t) + \frac{\Gamma}{N} \sum_j s_j (t)
% \end{equation}

% \lipsum[2]:

% \begin{figure}
% 	\centering
% 	\includegraphics[scale = 0.6]{images/helloWorld.png}
% 	\caption{
% 		\textbf{happy life}:
%         ascdkjahdc asdcksab dckajshdclsahd cakjshbd cka. 
% 	}
% 	\label{fig:label}
% \end{figure}

% \bibliographystyle{plain}
% \bibliography{main}
	
% \end{document}
